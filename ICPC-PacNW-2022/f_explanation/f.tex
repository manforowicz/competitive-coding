\documentclass{article}
\author{Marcin Anforowicz}

\begin{document}
In f.cpp (Food Processor), logarithms are used to determine how long it takes for a blade to cut down food to a given size. Here's how I derived the formula used.


First I defined some variables:
\begin{itemize}
\item $h$ - halving time
\item $a$ - starting size
\item $b$ - ending size
\item $t$ - time elapsed
\end{itemize}

And put them in this relation:

$$ b = a \cdot \left( \frac{1}{2} \right) ^{t/h} $$

Then I solved for t:

$$ \frac{b}{a} = \left( \frac{1}{2} \right) ^{t/h} $$

$$ \frac{t}{h} = \log_{1/2} \left( \frac{b}{a} \right) $$

$$ t = h \cdot \log_{1/2} \left( \frac{b}{a} \right) $$

And finally, I used the change of base formula:

$$ t = h \cdot \frac{\log \left( \frac{b}{a} \right)}{\log(1/2)} $$

This formula says how long it takes to get from one size to another given a certain halving time.

\end{document}